\documentclass[onecolumn,preprint,superscriptaddress,nofootinbib,notitlepage,10pt,linenumbers]{revtex4-1}

\usepackage[perpage]{footmisc}
\usepackage{epstopdf}
\usepackage{epsfig}
\usepackage{color}
\usepackage{hyperref}
\usepackage{mathbbol}
%\usepackage{bookmark}
\usepackage{tikz}
\usetikzlibrary{matrix,shapes,arrows,positioning,chains}
\usepackage{dutchcal}
\usepackage{calligra}

\DeclareMathAlphabet{\mathcalligra}{T1}{calligra}{m}{n}
\DeclareFontShape{T1}{calligra}{m}{n}{<->s*[2.2]callig15}{}

\usepackage{mathtools}
\usepackage{bbold}
\usepackage{amsmath,array}
\usepackage{amssymb}
\usepackage{slashed}
\usepackage{nicefrac}
\usepackage{tikz-feynman}
\usepackage{cjhebrew}
\usepackage{enumitem}
\usepackage{listings}
\lstset{language=[90]Fortran,
  basicstyle=\ttfamily,
  keywordstyle=\color{red},
  commentstyle=\color{green},
  morecomment=[l]{!\ }% Comment only with space after !
}

\setlength{\topmargin}{-1.0cm}
\setlength{\headheight}{0.1cm} \setlength{\footskip}{1.cm}
\setlength{\headsep}{.1cm}
\setlength{\textheight}{.8\paperheight}
\setlength{\textwidth}{.8\paperwidth}
\setlength{\oddsidemargin}{-.5cm}
\setlength{\evensidemargin}{0.0cm}
\setlength{\marginparwidth}{0.0cm}
\setlength{\marginparsep}{0.0cm}

\definecolor{blue}{HTML}{4169E1}
\definecolor{red}{HTML}{DC143C}
\definecolor{green}{HTML}{2E8B57}
\definecolor{black}{HTML}{000000}
\definecolor{g1}{HTML}{A9A9A9}
\definecolor{g2}{HTML}{696969}
\definecolor{g3}{HTML}{7F7F7F}
\definecolor{g4}{HTML}{D3D3D3}

\newcommand{\xdownarrow}[1]{%
  {\left\downarrow\vbox to #1{}\right.\kern-\nulldelimiterspace}
}

\newcommand*{\mprime}{^{\prime}\mkern-1.2mu}
\newcommand*{\mdprime}{^{\prime\prime}\mkern-1.2mu}
\newcommand*{\mtprime}{^{\prime\prime\prime}\mkern-1.2mu}
\newcommand*{\ethr}{e_\text{th}}
\newcommand{\black}[1]{\textcolor{black}{#1}}
\newcommand{\red}[1]{\textcolor{red}{#1}}
\newcommand{\blue}[1]{\textcolor{blue}{#1}}
\newcommand{\green}[1]{\textcolor{green}{#1}}
\newcommand{\caf}{\text{\cjRL{b}}}
\newcommand{\he}{${}^4$He}
\newcommand{\hes}{${}^3$He}
\newcommand{\tr}{${}^3$H}
\newcommand{\ls}{\ve{L}\cdot\ve{S}}
\newcommand{\eps}{\epsilon}
\newcommand{\cfg}{\texttt{cfg}}
\newcommand{\bv}{\texttt{bv}}
\newcommand{\as}{a_s}
\newcommand{\at}{a_t}
\newcommand{\ecm}{E_\textrm{\small c.m.}}
\newcommand{\dq}{\mbox{d\hspace{-.55em}$^-$}}
\newcommand{\mpis}{$m_\pi=137~${\small MeV}}
\newcommand{\mpim}{$m_\pi=450~${\small MeV}}
\newcommand{\mpil}{$m_\pi=806~${\small MeV}}
\newcommand{\muh}{\mu_{^3\text{\scriptsize He}}}
\newcommand{\mut}{\mu_{^3\text{\scriptsize H}}}
\newcommand{\mud}{\mu_\text{\scriptsize D}}
\newcommand{\pode}{\beta_{\text{\scriptsize D},\pm1}}
\newcommand{\poh}{\beta_{^3\text{\scriptsize He}}}
\newcommand{\pot}{\beta_{^3\text{\scriptsize H}}}
\newcommand{\com}[1]{{\scriptsize \sffamily \bfseries \color{red}{#1}}}
\newcommand{\eg}{\textit{e.g.}\;}
\newcommand{\ie}{\textit{i.e.}\;}
\newcommand{\cf}{\textit{cf.}\;}
\newcommand{\be}{\begin{equation}}
\newcommand{\ee}{\end{equation}}
\newcommand{\la}{\label}
\newcommand{\ber}{\begin{eqnarray}}
\newcommand{\eer}{\end{eqnarray}}
\newcommand{\nn}{\nonumber}
\newcommand{\half}{\frac{1}{2}}
\newcommand{\thalf}{\nicefrac[]{3}{2}}
\newcommand{\bs}[1]{\ensuremath{\boldsymbol{#1}}}
\newcommand{\bea}{\begin{eqnarray}}
\newcommand{\eea}{\end{eqnarray}}
\newcommand{\beq}{\begin{align}}
\newcommand{\eeq}{\end{align}}
\newcommand{\bk}{\bs k}
\newcommand{\bt}{B_{^{3}\text{H}}}
\newcommand{\bh}{B_{^{3}\text{He}}}
\newcommand{\bd}{B_\text{D}}
\newcommand{\ba}{B_\alpha}
\newcommand{\rgm}{$\mathbb{R}$GM}
\newcommand{\ev}[1] {|\bra #1  \ket |^2}
\newcommand{\parg}[1] {\paragraph*{-\,\textit{#1}\,-}}
\newcommand{\nopi}{\pi\hspace{-6pt}/}
\newcommand{\ve}[1]{\ensuremath{\boldsymbol{#1}}}
\newcommand{\xvec}{\bs{x}}
\newcommand{\rvec}{\bs{r}}
\newcommand{\sgve}{\ensuremath{\boldsymbol{\sigma}}}
\newcommand{\tave}{\ensuremath{\boldsymbol{\tau}}}
\newcommand{\na}{\nabla}
\newcommand{\bra}[1] {\left\langle~#1~\right|}
\newcommand{\ket}[1] {\left|~#1~\right\rangle}
\newcommand{\bet}[1] {\left|#1\right|}
\newcommand{\overlap}[2] {\left\langle\,#1\,\left|\,#2\,\right.\right\rangle}
\newcommand{\me}[3] {\left\langle\,#1\,\left|\left.\,#2\,\right|\,#3\,\right.\right\rangle}
\newcommand{\redme}[3] {\left\langle\,#1\,\middle|\right|\,#2\,\left|\middle|\,#3\,\right\rangle}
\newcommand{\lam}[1]{$\Lambda=#1~$fm$^{-1}$}\newcommand{\tx}{\tilde{x}}
\newcommand{\eftnopi}{\mbox{EFT($\slashed{\pi}$)}}
\newcommand{\threej}[6]{\ensuremath{\begin{pmatrix}#1 & #2 & #3\\#4&#5&#6 \end{pmatrix}}}
\newcommand{\clg}[6]{\ensuremath{\left\langle\left.\,#1#2\,;\,#3#4\,\right\vert\right.\left.\,#5#6\,\right\rangle}}
\newcommand{\re}[1] {\mathcal{R}\left[#1\right]}
\newcommand{\im}[1] {\mathcal{I}\left[#1\right]}
\newcommand{\E}{\mathcal{E}}
\newcommand{\eqr}[1] {Eq.~\eqref{#1}}
\newcommand{\op}[1] {\hat{\mathcal{O}}_{\text{\small #1}}}

\usepackage{pifont}
\renewcommand\thefootnote{\ding{\numexpr171+\value{footnote}}}
\let\endtitlepage\relax

\begin{document}

\title{Lorentz and Siegert offer Compton their assistance.}
\author{Jean Luc Picard}
\email{jeanluc@1701.ncc}
\affiliation{Starfleet Academy, Fort Baker, San Francisco, Earth}
\date{\today}

\begin{abstract}
We study (elastic) Compton scattering off nuclei as a function of the
nucleon number. We consider processes with an initial state comprised of
photons, pions, protons, and neutrons at energies which are insufficient to
create particles that are not composites of these basic degrees of freedom.
The appropriate theory for amplitudes at momentum scales of the order of the pion mass
is constrained by Galilean, chiral, and $U(1)$-gauge symmetries.
The systematic ordering of the interactions thence admissible, allows us to
resolve the various correlations between observables in systems of different
size, kinematics, level of excitation, {\it etc.}. Thereby, we investigate to what
extent properties of large systems can be understood as consequences of the
two- and three-body interaction, but we also address the practical problem, \eg,
to assess the induced uncertainty in a few-body observable due to that in a
one-body quantity -- {\it To what extent does our poor knowledge about the neutron
polarizability affect predictions of deuteron and helium properties?}

An intriguing problem in its own right is the correct application of the
coupling of the two photons to the hadrons and the meson as specified and well
understood in {\it chiral perturbation theory} -- a relativistic quantum field theory --
in a solution of the non-relativistic few-nucleon problem.
\end{abstract}

%\maketitle


\paragraph{The ``LIT equation''}
\be\la{eq.liteq}
\left(\hat{H}_\text{nuclear}\underbrace{-E_0-\re{\sigma}-i~\im{\sigma}}_{:=-\E}\right)\Psi_\text{LIT}^{J^\pi m_j}=
\left[\hat{\mathcal{O}}_{Lm_L}\left\lbrace\bet{\ve{k}},\ve{j}_{v}\right\rbrace\otimes\Psi_0^{J_0^{\pi_0}}\right]^{J^\pi m_j}\;\;.
\ee
with
\begin{align}\la{eq.liteq.descr}
v(\text{ertex})\in&\left\lbrace~\ve{j}_o(\ve{x})=\ldots~,~\ve{j}_s(\ve{x})=\ldots~,~\ve{j}_{mec}(\ve{x})=\ldots~,~\ldots\right\rbrace\;\;;
\end{align}

\paragraph{The variational basis}
\be\la{eq.basis}
\Psi_\text{LIT}^{J^\pi m_j}=\sum\limits_nu_n~\phi^{J^\pi m_j}_n\;\;.
\ee
with
\begin{align}\la{eq.basis.descr}
\phi^{J^\pi m_j}_n\in\left\lbrace~\left.
\left[\xi_{S_n}\otimes\mathcal{Y}_{l_n}(\ve{\rho})\right]^{J m_j}~
e^{-\gamma_n\ve{\rho}^2}~
\right\vert~\gamma\in\mathbb{R}_+~,~s\in\mathbb{N}^{A-1}+\left(\frac{\mathbb{N}}{2}\right)^{A-2}~,~l\in\mathbb{N}^{A-1}+\mathbb{N}^{A-2}~\right\rbrace
\end{align}

\paragraph{The matrix form of the ``LIT equation''}
\be\la{eq.liteq.mat}
\sum\limits_{s=1}^{N_\text{LIT}}\phi^{J^\pi m_j}_r\left(\hat{H}_\text{nuclear}-\E\right)\phi^{J^\pi m_j}_s~u_s=
\sum\limits_{n=1}^{N_0}\sum\limits_{m_L}~c_n~\underbrace{\clg{L}{J_0}{m_L}{m_j-m_L}{J}{m_j}}_{\text{Eq.\eqref{eq.litinhomo}}}~\phi^{J^\pi m_j}_r\hat{\mathcal{O}}_{Lm_L}\,\phi^{J_0^{\pi_0}(m_j-m_L)}_n\;\;.
\ee
with
\begin{align}\la{eq.liteq.mat.descr}
N_\text{LIT}~:&~\text{number of basis states used to expand the LIT state, \eg,} \Psi_\text{LIT}^{2^-}\;\;\;\;;\\
N_0\leq N_\text{LIT}~:&~\text{number of basis states used to expand the target, \eg, the deuteron;}\\
\end{align}

\paragraph{The matrix element}
Reduced matrix elements are used where possible in order to reduce the number of numerical evaluations.
The composite structure of the operators (spatial and spin form a scalar) demands a more general version
of the Wigner-Eckart theorem (\cf\cite{edmonds_book}, Eq.(7.1.5)).
\begin{align}\la{eq.me}
\phi^{J^\pi m_j}_m~\hat{\mathcal{O}}_{Lm_L}~\phi^{J_0^{\pi_0}m_{j_0}}_n:=&
\me{m;l_lS_lJ_lm_{j_l}}{\mathcal{A}~\mathcal{O}_{Lm_L}}{l_rS_rJ_rm_{j_r};n} \\
=&\underbrace{(-1)^{L-J_r+J_l}\frac{\clg{L}{J_r}{m_L}{m_{j_r}}{J_l}{m_{j_l}}}{\hat{J_l}}}_
{\blue{\texttt{enemb:600ff}}}
\underbrace{\redme{m;l_lS_lJ_l}{\mathcal{A}~\mathcal{O}_{L}}{l_rS_rJ_r;n}}_{\xdownarrow{.75cm}}\\
&\underbrace{\hat{J_r}\hat{J_l}\hat{L}
\left\lbrace\begin{array}{ccc}l^m_l & l^n_r & p\\ S^m_l & S^n_r & q \\ J_l & J_r & L\end{array}\right\rbrace}_{\blue{\texttt{enemb:ecce}}}
\sum_\text{dc}\sum_{\mathfrak{p}\in\text{dc}}
\underbrace{\redme{m;l^m_l}{\mathcal{O}^o_p}{\mathcal{A}_\text{dc}l^n_r;n}}_{\red{\texttt{luise}}}\cdot
\underbrace{\redme{m;S^m_l}{\mathcal{O}^s_q}{\mathcal{A}_\mathfrak{p}S^n_r;n}}_{\red{\texttt{obem}}}\\
=&~\hat{J_r}\hat{L}\clg{\green{J_r}}{\red{L}}{\green{m_{j_r}}}{\red{m_L}}{J_l}{m_{j_l}}\cdot
\left\lbrace\begin{array}{ccc}l^m_l & l^n_r & p\\ S^m_l & S^n_r & q \\ J_l & J_r & L\end{array}\right\rbrace\\
&\cdot\sum_\text{dc}\sum_{\mathfrak{p}\in\text{dc}}
\redme{m;l^m_l}{\mathcal{O}^o_p}{\mathcal{A}_\text{dc}l^n_r;n}\cdot
\redme{m;S^m_l}{\mathcal{O}^s_q}{\mathcal{A}_\mathfrak{p}S^n_r;n}
\end{align}
with
\begin{align}\la{eq.liteq.mat.descr}
\hat{a}:=\sqrt{2a+1}\;\;;\\
\mathcal{A}=\sum_{\mathfrak{p}\in\mathcal{S}_{A-1}}(-1)^{\text{sgn}(\mathfrak{p})}\hat{\mathfrak{p}}=\oplus_\text{dc}\\
\text{dc ~:~ double co-set}
\end{align}

\paragraph{The calculation}

\begin{enumerate}[label=(\roman*)]

\item Solve $$\hat{H}_\text{nuclear}~\Psi^{J_0^{\pi_0}}=E_0~\Psi^{J_0^{\pi_0}}$$ with the ansatz
$$\Psi^{J_0^{\pi_0}}=\sum\limits_nc_n~\phi^{J_0^{\pi_0}}_n\;\;\;.$$
If $\hat{H}_\text{nuclear}$ is a spherical rank-0 operator -- a condition which most practical nuclear potentials satisfy --
$\Psi^{J_0^{\pi_0}}\neq f(m_{j_0})$. We obtain $\Psi^{J_0\red{J_0}}$, in practice.

\item Calculate
$$\mathbb{H}_{rs}:=\me{\phi^{J^\pi}_r}{\hat{H}_\text{nuclear}}{\phi^{J^\pi}_s}\;\;\text{and}\;\;\mathbb{N}_{rs}:=\overlap{\phi^{J^\pi}_r}{\phi^{J^\pi}_s}\;\;\;
\forall~|L-J_0|\leq J\leq|L+J_0|$$

\item \blue{$\forall m_j\,\&\,m_L$}, calculate
$$S^{Jm_jJ_0}_{rs,m_L}:=\me{\phi^{J^\pi m_j}_r}{\hat{\mathcal{O}}_{Lm_L}}{\phi^{J_0^{\pi_0}J_0}_s}\;\;,$$
and superimpose these matrix elements according to Eq.\eqref{eq.liteq.mat}
\be\la{eq.litinhomo}
S^{Jm_j}_r:=\sum\limits_{m_L}~\clg{L}{J_0}{m_L}{\red{m_j}\black{-m_L}}{J}{m_j}~\underbrace{\sum\limits_{n}^{N_0}~c_n~S^{Jm_jJ_0}_{rn,m_L}}
_{\text{\bf enemb.f OUT}}\;\;\;.
\ee
Ecce, $\Psi^{J_0^{\pi_0}}\neq f(m_{j_0})$ does not allow for an elimination of $m_j$ from this equation!

\item Solve the (complex) linear matrix equation
\be\la{eq.liteq2}
\left(\mathbb{H}_{rs}-\E \mathbb{N}_{rs}\right)u^{Jm_j}_s=S^{Jm_j}_r
\ee
to obtain the LIT state
\be\la{eq.litstate}
\psi^{v(\text{ertex}),(\text{mu})L(\text{tipolarity})}_{J_{i(\text{nitial})/f(\text{inal})};J_{(\text{i})n(\text{termediate})}m_n}(k,\sigma)
=\blue{\psi^{v,L}_{J_0;Jm_j}(k,\sigma)}\black{:=
\Psi_\text{LIT}^{J^\pi m_j}\big(\underbrace{\bet{\ve{k}},v,L}_{\text{vertex}\atop\text{quantum numbers}};\underbrace{E_0,J_0}_{\text{initial/final-state}\atop\text{quantum numbers}};\re{\sigma},\im{\sigma}\big)\;\;\;.}
\ee

\item Units exemplified for the $E1$ calculation:
\begin{gather}
\intertext{The units of the LIT ($\psi$) and the ground state ($\phi$) are not identical, in general:}
\op{nucl}(\text{MeV})\psi(x)=\op{E1}(0)\phi(\text{MeV}^{3/2})
\intertext{$E1$ is implemented as $j_1Y_{1M}$ without any dimensionful prefactors, and hence}
x=\text{MeV}^{1/2}\;\;\Rightarrow\;\;[L]=\text{MeV}^{-2}
\intertext{To recover the physical $E1$, a factor of}
\frac{2}{\hbar k}
\intertext{multiplies the RHS.}
\end{gather}

\item The inner product
\begin{align}\la{eq.litME}
\mathcal{L}_{v'L',vL}^{J_f,J_i;J}(k',k,\sigma)=&(-1)^{J-J_i+L-L'+v'}N_{J,\sigma}\sum_{m_j}
\underbrace{\overlap{\psi^{v',L'}_{J_f;Jm_j}(k',\sigma)}{\psi^{v,L}_{J_i;Jm_j}(k,\sigma)}}_{=\sum_{r,s}(u^{Jm_j}_r)^*u^{Jm_j}_s~\mathbb{N}_{rs}}\\
=&\int_{\ethr}^\infty~\frac{\mathcal{F}_{v'L',vL}^{J_f,J_i;J}(k',k,\red{E})}{(E-\sigma)(E-\sigma^*)}~dE\la{eq.litMEb}
\intertext{with $N_{J,\sigma}$ being the multiplicity of Lorentz states
 for given $J$ and $\sigma$.}
\end{align}

\item The recovery of the (partial) strength functions --
 The inverse Lorentz-integral transformation

At this stage, the problem is to recover $\mathcal{F}$, given $\mathcal{L}$ by inverting
the integral transformation \eqr{eq.litMEb}.
The {\it discrete essence} of that can be written as
\be
\ve{L}_s=\sum_eA_{se}\ve{F}^e=\sum_nc_n\sum_eA_{se}f_n^e\;\;.
\ee
If the energy sum/integral can be evaluated efficiently for a set of basis vectors $f_n$
which is suitable for a proper representation of the vector $\ve{L}$, it remains to solve
a linear optimization problem. This is detailed below from \eqr{eq.responseexp} onwards.

The choice for the basis $f_n$ is based on two ideas. Firstly, from the definition
\be\la{eq.partstrength}
\mathcal{F}_{v'L',vL}^{J_f,J_i;J}(k',k,\red{E})=\underbrace{N_{J,E}}_{\#J,E-\text{states}}
\redme{J_f~E_f}{M^{\nu\mprime,L\mprime}(k\mprime)}{J~E}\,\redme{J~E}{M^{\nu,L}(k)}{J_i~E_i}\;\;,
\ee
the joint probability to, first, induce via $M^{\nu,L}(k)$ the transition from one eigenstate of
a system to another, which, second is perturbed via $M^{\nu\mprime,L\mprime}(k\mprime)$ into
some final eigenstate. Let me imagine a system with one localized bound state. Its wave
function is folded with the perturbation, which is a polynomial in $k$ of order $L$ -- the
multipolarity -- and the wave function of another eigenstate of the system, constrained
by energy conservation. I think of $k$ large enough such that the latter state contains a
free wave on some coordinate. Hence the matrix element retains a Fourier-transform character
of a localized wave packet multiplied by a polynomial. The latter can be expanded well
in symmetrical, peaked functions -- \eg, Gaussians -- and the polynomial skews those.
A Fourier transform of such a skewed Gaussian will retain its shape but become
broader/narrower. A skewed Gaussian is expected to be amenable to an expansion in
Lorentz functions.

The second, related idea for choosing a Lorentz basis is the ability to calculate
the integral transform with a Lorentz kernel analytically (viz. \eqr{eq.ltofl}).

\end{enumerate}

\newpage

\section{formulas and constants}

\begin{alignat}{2}
\text{(Wigner) 3-$j$ symbol:} &\hspace{1cm}& \threej{L}{S}{J}{m_l}{m_s}{-m_j}=&~(-1)^{L-S+m_j}~(2J+1)^{-\frac{1}{2}}~\clg{L}{S}{m_l}{m_s}{J}{m_j}\\
\text{Matrix for single-axis rotation:} &\hspace{1cm}&   \mathcal{D}_{m\mprime,m}^{(j)}(0~\beta~0)\equiv&~d_{m\mprime,m}^{(j)}(\beta)\nonumber\\
&\hspace{1cm}& =&\left[\frac{(j+m\mprime)!(j-m\mprime)!}{(j+m)!(j-m)!}\right]^\frac{1}{2}\nonumber\\
&\hspace{1cm}&&~\cdot\sum_\sigma\left(j+m\atop j-m\mprime-\sigma\right)\left(j-m\atop \sigma\right)(-1)^{j-m\mprime-\sigma}\nonumber\\
&\hspace{1cm}&&~\cdot\left(\cos\frac{\beta}{2}\right)^{2\sigma+m+m\mprime}\left(\sin\frac{\beta}{2}\right)^{2j-2\sigma-m-m\mprime}\\
\end{alignat}

\paragraph{The Lorentz transformation of a Lorentzian\\}\la{sec.LoOfLo}

The strength functions which constitute
the Compton amplitudes are themselves composed of scalar functions of an energy parameter.
I assume that these functions can be expanded to any desired accuracy in a Lorentzian/Cauchy
basis as follows:

\be\la{eq.responseexp}
r(e)=\sum_nc_n\frac{\theta(e-\ethr)}{(a_n-e)^2+b_n^2}:=\sum_nc_n\theta(e-\ethr)f_n~.
\ee

As the process is non-trivial only if the photon's energy exceeds a minimum threshold
energy $\ethr$, it is in order to include a system-characteristic step function.

In practice, we obtain an integral transformation of this quantity with a Lorentzian kernel,

\be
L(\sigma):=\int\limits^\infty_{-\infty}\,\frac{r(e)}{(\sigma_r-e)^2+\sigma_i^2}~de
=\sum_nc_n\int\limits^\infty_{e_\text{th}}\,\frac{f_n(e)}{(\sigma_r-e)^2+\sigma_i^2}~de~.
\ee

If basis basis functions\footnote{$x\mprime:=x-\ethr$} which we define through the integral

\begin{eqnarray}\la{eq.ltofl}
\int^\infty_{\ethr}\,\frac{f_n(e)}{(\sigma_r-e)^2+\sigma_i^2}~de
&=&\int_{\ethr}^\infty\,\frac{1}{(\sigma_r-e)^2+\sigma_i^2}\cdot
\frac{1}{(a_n-e)^2+b_n^2}~de\nonumber\\
&\vdots&\nonumber\\
&=&\left[(\sigma_r-a_n)^2+(\sigma_i-b_n)^2\right]^{-1}\cdot
\left[(\sigma_r-a_n)^2+(\sigma_i+b_n)^2\right]^{-1}\nonumber\\
&&\cdot
\Bigg\lbrace
\sigma_i^{-1}\left((\sigma_r-a_n)^2+b_n^2-\sigma_i^2\right)
\left(\frac{\pi}{2}+\tan^{-1}\left(\frac{\sigma_r\mprime}{\sigma_i}\right)\right)\nonumber\\
&&+~
b_n^{-1}\left((\sigma_r-a_n)^2-b_n^2+\sigma_i^2\right)
\left(\frac{\pi}{2}+\tan^{-1}\left(\frac{a_n\mprime}{b_n}\right)\right)\nonumber\\&&+~
(\sigma_r-a_n)~\ln\left(\frac{\sigma_r\mprime^2+\sigma_i^2}{a_n\mprime^2+b_n^2}
\right)
\Bigg\rbrace\nonumber\\
&:=&L_n(\sigma,\ethr)
\end{eqnarray}

allow for an accurate expansion of a given, physical function $L(\sigma)$,
namely for a set $\mathfrak{B}=\left\lbrace (a_n,b_n)\in\mathbb{R}_{0^+}\right\rbrace_{n=1,\ldots,d}$,
we seek

\be
\min_{\ve{c}}\left\vert L(\sigma)-\sum_nc_nL_n(\sigma,\ethr)\right\vert:=\ve{c}^*.
\ee

The set of optimal parameters $\ve{c}^*$ represents via Eq.~\eqref{eq.responseexp} an expansion
of the response function in Lorentzians.
If $\mathfrak{B}$ is ``numerically'' complete, the dependence of $r(e)$ on changes
in this basis should be negligible, \ie, the problem is {\bf not} ill-posed.
This shall now be demonstrated for an exemplary process.



\begin{table}
\be\la{tab.constants}
\renewcommand{\arraystretch}{2.4}
\setlength{\tabcolsep}{8pt}
\begin{array}{c|c|c}\hline
\alpha=\frac{e^2}{4\pi}&\text{dimensionless}&\frac{1}{137.03604}\\
\hbar c & & 197.32858~\text{MeV}\cdot\text{fm}^2 \\
\hline\hline
\end{array}
\ee
\caption{Implemented numerical values.}
\end{table}

\section{``How to do an integral''}

\paragraph{The multipole operators}
\begin{itemize}
  \item convection current\footnote{Indices referring to particles are put in brackets.}
\begin{subequations}
  \begin{align}
  \ve{j}_o(\ve{x})=&\frac{e}{2m}\sum\limits_i^{A}\frac{1}{2}\left(1+\tau_z(i)\right)
  \left\lbrace\ve{p}(i)~,~\delta^{(3)}(\ve{x}-\ve{r}(i))\right\rbrace\la{eq.conc}\\
  \text{\bf(electric)}~
   \mathcal{O}_{Lm_L}^{o^{\text{el}}}=&\frac{e\hbar}{mc}\hat{L}^{-1}\sum\limits_i^Ag_l(i)\left[\sqrt{L}\Delta_{LM}^{L+1}(\ve{r}(i))-\sqrt{L+1}\Delta_{LM}^{L-1}(\ve{r}(i))\right]\la{eq.convcel}\\
   \text{\bf(magnetic)}~
\mathcal{O}_{Lm_L}^{o^{\text{mag}}}=&i\frac{e\hbar}{mc}\sum\limits_i^Ag_l(i)\Delta_{LM}^L(\ve{r}(i))\la{eq.convcmag}
  \end{align}
\end{subequations}
  \item spin current
\begin{subequations}
  \begin{align}
  \ve{j}_s(\ve{x})=&\frac{e\hbar}{2m}\sum\limits_i^{A}\frac{1}{2}\left(g_{s_p}\left(1+\tau_z(i)\right)+g_{s_n}\left(1-\tau_z(i)\right)\right)\ve{\sigma}(i)\times\ve{\nabla}(i)\delta^{(3)}(\ve{x}-\ve{r}(i))\la{eq.spinc}\\
  \text{\bf(electric)}~
\mathcal{O}_{Lm_L}^{s^{\text{el}}}=&-\frac{e\hbar\red{|\ve{k}|}}{2mc}\sum\limits_i^Ag_s(i)\sum\limits_{M,\nu}\clg{L}{1}{M}{\nu}{L}{m_L}\cdot\ve{\sigma}_\nu(i)\Phi_{LM}(\ve{r}(i))\la{eq.spincel}\\
\text{\bf(magnetic)}~
\mathcal{O}_{Lm_L}^{s^{\text{mag}}}=&i\frac{e\hbar\red{|\ve{k}|}}{2mc}\hat{L}^{-1}\sum\limits_i^Ag_s(i)\sum\limits_{M,\nu}
\left[\sqrt{L}\clg{L+1}{1}{M}{\nu}{L}{m_L}\ve{\sigma}_\nu(i)\Phi_{L+1M}(\ve{r}(i))\right.\\
&\left.-\sqrt{L+1}\clg{L-1}{1}{M}{\nu}{L}{m_L}\ve{\sigma}_\nu(i)\Phi_{L-1M}(\ve{r}(i))\right]
\la{eq.spincmag}
\end{align}
\end{subequations}
\end{itemize}
with
\begin{align}\la{eq.multipoleop.descr}
\Phi_{Lm_L}(\ve{r})=&j_L(kr)~Y_{Lm_L}(\Omega_r)&&&\footnotemark\\
\Delta_{Lm_L}^J(\ve{r})=&\sum\limits_{M,\nu}\clg{L}{1}{M}{\nu}{J}{m_L}\Phi_{Lm_L}(\ve{r})~\ve{p}_\nu\;\;.&&&\footnotemark
\end{align}
\addtocounter{footnote}{-1}
\footnotetext{rank-L spherical $\ve{L}^2$ tensor}
\stepcounter{footnote}
\footnotetext{linear combination of spherical $\ve{L}^2$ rank-L tensors, hence, itself a rank-L spherical $\ve{L}^2$ tensor and \red{not} 
rank-J spherical $\ve{L}^2$.}

\paragraph{Siegert form}
Linear-in-$\alpha$ coupling of a (background) photon (momentum $\ve{k}$) to a system ($j^\nu$) at $\ve{x}$:
\begin{align}\la{eq.siegert}
\me{f}{-\frac{1}{c}\int d\ve{x}A_\nu(\ve{x})j^\nu(\ve{x})}{i}&\stackrel{\text{transverse}}{\longrightarrow}&\me{f}{-\sum_{LM}i^L\hat{L}\sqrt{2\pi}
\left(\mathcal{O}_{Lm_L}^{{\text{el}}}+\mu\mathcal{O}_{Lm_L}^{{\text{mag}}}\right)}{i}\\
&\stackrel{\text{el. only}}{\longrightarrow}&-\sum_{LM}i^L\hat{L}\sqrt{2\pi}\me{f}{\left(\frac{1}{ck}\int d\ve{x}~\ve{j}(\ve{x})\cdot\ve{\nabla}_x\times\ve{L}\left[j_L(kx)Y_{LM}(\Omega_x)\right]\right)}{i}
\\[1em]
&\stackrel{k\to 0}{\longrightarrow}&~-\sum_{LM}\frac{i^{L+1}\hat{L}\sqrt{2\pi}}{k}\cdot\frac{L+1}{L}\me{f}{\left(\int d\ve{x}~\ve{j}(\ve{x})\cdot\ve{\nabla}_x\left[j_L(kx)Y_{LM}(\Omega_x)\right]\right)}{i}
\\[1em]
&=&~\sum_{LM}i^L\cdot\frac{\hat{L}\sqrt{2\pi}}{\hbar k}\cdot\frac{L+1}{L}\me{f}{\int d\ve{x}~\left[\,\rho(\ve{x})\,,\,\hat{H}_\text{nuclear}\,\right]~j_L(kx)Y_{LM}(\Omega_x)}{i}\\[1em]
&\overset{\scriptstyle {L=1\atop \rho=\rho^{(1)}}}{=}&~\frac{2}{\hbar k}\me{f}{\sum\limits_i^A~q(i)~\left[\,j_1(kr(i))Y_{1M}(\Omega_{\ve{r}(i)})\,,\,\hat{H}_\text{nuclear}\,\right]}{i}
\end{align}
with
\begin{align}\la{eq.siegert.descr}
\ve{L}=&-i\hbar\left(\ve{x}\times\ve{\nabla}_x\right)\\
\rho^{(1)}(\ve{x})=&\sum\limits_i^A\underbrace{\frac{e}{2}(1+\tau_z(i))}_{:=q(i)}\delta^{(3)}(\ve{x}-\ve{r}_i)\\
\lim\limits_{x\to 0}j_l(x)=&\frac{x^l}{(2l+1)!!}\;\;.
\end{align}
We abstain 

\paragraph{The non-trivial matrix element (which serves {\bf 2-body} currents, too)}
\be\la{eq.theme}
\me{m;l_lm_{l_l}}{\Phi_{Lm_L}(\ve{\rho_\nu})~e^{-\beta\ve{r}_{ij}}\prod\limits_{N}^{N_{\text{op}}}\mathcal{Y}_{L_NM_N}(\ve{r}_{ij})}{l_rm_{l_r};n}
\ee

\newpage
\section{Obtaining a cross section}
If one is interested in a quantification of the reaction of a nucleus being irradiated with
an electromagnetic wave, the following formulae might help.

\begin{align}
\intertext{
  For the total cross section for the reaction of a system initially in state $\alpha$
  induced by a perturbation which transforms $\alpha$ into a set of final states (labeled $\alpha'$)
  Taylor (\cf\cite{taylor_book}~Eq.(17.17)) writes
}
\sigma(\star\gets\alpha)=&~\sum_{\alpha'}\sigma(\alpha'\gets\alpha)\notag\\
=&~(2\pi)^4~\frac{m}{k}~\sum_{\alpha'}\int d\ve{k}'~\delta(E'-E)\left\vert t(\ve{k}',\alpha'\gets\ve{k},\alpha)\right\vert^2\quad.\quad
\intertext{The perturbation is implicit in the transition matrix $t$. For an electromagnetic current,}
\end{align}
\newpage
\section{Notes on the \texttt{RRGM} implementation}

\subsection*{The expansion of the 3-helium ground state}
We approximate the antisymmetric nuclear three-body state with the expansion

\begin{eqnarray}
\overlap{\ve{\rho}_1,\ve{\rho}_2}{^3\text{He}}&=&\sum_i~c_i \, e^{-\gamma_i\ve{\rho}_1^2} \, e^{-\delta_i\ve{\rho}_2^2}\cdot
\left[\left[\mathcal{Y}_{l_{1,i}}(\ve{\rho}_1)\otimes\mathcal{Y}_{l_{2,i}}(\ve{\rho}_2)\right]^{L_i}\otimes
\left[\left[\xi_1\otimes\xi_2\right]^{\mathcalligra{s}_i}\otimes\xi_3\right]^{S_i}\right]^J\\
&\equiv&\sum_i~c_i~\phi_i(\ve{\rho}_1,\ve{\rho}_2)\nonumber
\;\;.
\end{eqnarray}

A single basis vector is parametrized by two width parameters $\gamma$ and $\delta$, and a (spin) angular-momentum coupling scheme
which specifies two angular-momentum quanta $l_{1,2}$ (one for each Jacobi coordinate $\ve{\rho}_{1,2}$), a total orbital
angular momentum $L$, an intermediate spin coupling $\mathcalligra{s}$, and the total spin $S$.

In practise, we proceed as follows:

\begin{enumerate}
\item Fix an orbital-angular-momentum cutoff $l_\text{max}$ and add {\bf all} spin- and orbital coupling schemes
to the basis with $l_{1,2}\leq l_\text{max}$ and intermediate/total angular momenta which contribute to a total $J$ and parity.

The structure $\left[\left[\mathcal{Y}_{1}(\ve{\rho}_1)\otimes\mathcal{Y}_{1}(\ve{\rho}_2)\right]^{2}\otimes
\left[\left[\xi_1\otimes\xi_2\right]^{1}\otimes\xi_3\right]^{\nicefrac{3}{2}}\right]^{\nicefrac{1}{2}^+}$, for example, would
be included if $l_\text{max}=2$ and combines two negative-parity solid harmonics to form a of positive parity.
\item For each of the above blocks, two sets of width parameters are chosen. At present, this means selecting upper and lower
bounds $\gamma/\delta_{\text{min,max}}$ of geometric grids with a fixed number of nodes $n$.

For a particular coupling block, one uses
\mbox{$\gamma\in\lbrace\gamma_\text{min},\gamma_\text{max}\rbrace^{n(l_{1,2,i},L_i,\mathcalligra{s}_i,S_i)}_\text{geom}\equiv w_\gamma$} and\\
\mbox{$\delta\in\lbrace\delta_\text{min},\delta_\text{max}\rbrace^{m(l_{1,2,i},L_i,\mathcalligra{s}_i,S_i)}_\text{geom}\equiv w_\delta$}.
Hence, for each coupling block there are $m\cdot n$ (``in principle'') independent Gaussian prefactors. To select those combinations
which form a numerically stable and yet complete-for-its-purpose set is an unsolved problem, if the state should be fed into a LIT
calculation.
\end{enumerate}

\newpage

\subsection*{genetic basis selection}

\begin{description}
  \item[initial basis -- {\it the first generation of parents}~]${}$\vspace{.2cm}\\
A single basis vector is parametrized by a particular orbital- and (iso)spin angular-momentum coupling scheme,
$\cfg_j:=\lbrace l^{(j)}_{1,2},L^{(j)},\mathcalligra{s}^{(j)},S^{(j)},\mathcalligra{t}^{(j)},T^{(j)}\rbrace$, {\bf and}
a pair of width parameters, $\bv_i:=\lbrace\gamma^{(i)},\delta^{(i)}\rbrace$.

We obtain an initial set of vectors by defining {\bf upper} and {\bf lower} bounds within which logarithmically spaced
values are selected {\bf for each} \cfg. As $$\overlap{\cfg_n\bv_i}{\cfg_m\bv_i}\propto\delta_{mn}\qquad,$$
\ie, choosing a basis which contains elements which differ solely in their discrete quantum numbers but have {\bf identical}
widths, appears not to introduce linearly dependent vectors, the (anti)symmetrization of the basis,
$$\mathbb{1}:=\me{\cfg_n\bv_i}{\mathcal{\hat{A}}}{\cfg_m\bv_j}\qquad,$$

\end{description}

\newpage



\newpage

\tikzset{
desicion/.style={
    diamond,
    draw,
    text width=4em,
    text badly centered,
    inner sep=0pt
},
block/.style={
    rectangle,
    draw,
    text width=20em,
    text centered,
    rounded corners
},
cloud/.style={
    draw,
    ellipse,
    minimum height=2em
},
descr/.style={
    fill=white,
    inner sep=2.5pt
},
connector/.style={
    -latex,
    font=\scriptsize
},
rectangle connector/.style={
    connector,
    to path={(\tikztostart) -- ++(#1,0pt) \tikztonodes |- (\tikztotarget) },
    pos=0.5
},
rectangle connector/.default=-2cm,
straight connector/.style={
    connector,
    to path=--(\tikztotarget) \tikztonodes
}
}

\begin{tikzpicture}
\matrix (m)[matrix of nodes, column  sep=2cm,row  sep=8mm, align=center, nodes={rectangle,draw, anchor=center} ]{
    |[block]| {Def. $\mathbb{B}_0$, \eg, $P^x_\text{Laplace}=\frac{1}{b}~e^{-|x-\mu|/b}$}               &  \\
    |[block]| {Remove all BV from $\mathbb{B}_0$ with insignificant
     contribution to $\Psi_0$}               &  $\gamma_{\text{clean}}$ &                                             \\
    |[block]| {Def. $\mathbb{B}_\text{LIT}$, \eg, $P^x_\text{log}=\frac{\log x}{b(\log b-1)-a(\log a-1)}$}               &  \\
    |[block]| {Remove all BV from $\mathbb{B}_\text{LIT}$ with $|\gamma_\text{LIT}-\gamma_\text{clean}|<\epsilon$}               &  $\gamma_{\text{LIT}}$ &                                             \\     
};
\path [>=latex,->] (m-1-1) edge (m-2-1);
\path [>=latex,->] (m-2-1) edge (m-2-2);
\path [>=latex,->] (m-3-1) edge (m-4-1);
\path [>=latex,->] (m-4-1) edge (m-4-2);

\end{tikzpicture}

\subsection*{luise.f~$\to$~qual.f}
$P_{dc}$ and width-independent quantities of $\Gamma_{l_1m_1,\ldots,l_zm_z}$ with
$z=n_{c_l}-1+n_{c_r}-1+n_{ww}$.

\subsection*{obem.f~$\to$~qual.f}


\subsection*{qual.f~$\to$~enemb.f}

\begin{lstlisting}
 WRITE(NBAND1) NZF,MUL,(LREG(K),K=1,NZOPER),I,(NZRHO(K),K=1,NZF)

 WRITE(NBAND1) N3,MMASSE(1,N1,K),MMASSE(2,N1,K),MLAD(1,N1,K),
1 MLAD(2,N1,K),MSS(1,N1,K),MSS(2,N1,K),MS(N1,K),
2 (LZWERT(L,N2,K),L=1,5),(RPAR(L,K),L=1,N3),KP(MC1,N4,K)

 WRITE(NBAND1) NTE,NC,ND,ITV2

 WRITE(NBAND1) ((IND(MM,NN), NN=1, JRHO), MM=1, IRHO)

 WRITE(NBAND1) NUML, NUMR, IK1H, JK1H, LL1,
*               ((F(K,L),(J-1,DM(K,L,J),J=1,LL1),L=1,JK1),
*                                                K=1,IK1)
\end{lstlisting}

\nocite{*}
\bibliography{refs_compton}

\end{document}