\documentclass[onecolumn,preprint,superscriptaddress,nofootinbib,notitlepage,10pt,linenumbers]{revtex4-1}

\usepackage[perpage]{footmisc}
\usepackage{epstopdf}
\usepackage{epsfig}
\usepackage{color}
\usepackage{hyperref}
\usepackage{mathbbol}
%\usepackage{bookmark}
\usepackage{tikz}
\usetikzlibrary{matrix,shapes,arrows,positioning,chains,backgrounds}
\usepackage{dutchcal}
\usepackage{calligra}

\DeclareMathAlphabet{\mathcalligra}{T1}{calligra}{m}{n}
\DeclareFontShape{T1}{calligra}{m}{n}{<->s*[2.2]callig15}{}

\usepackage{mathtools}
\usepackage{bbold}
\usepackage{amsmath,array}
\usepackage{amssymb}
\usepackage{amsfonts}

\usepackage{tabularx}
\usepackage{slashed}
\usepackage{nicefrac}
\usepackage{tikz-feynman}
\usepackage[framemethod=TikZ]{mdframed}
\usepackage{cjhebrew}
\usepackage{enumitem}
\usepackage{listings}
\lstset{language=[90]Fortran,
  basicstyle=\ttfamily,
  keywordstyle=\color{red},
  commentstyle=\color{green},
  morecomment=[l]{!\ }% Comment only with space after !
}

\setlength{\topmargin}{-1.0cm}
\setlength{\headheight}{0.1cm} \setlength{\footskip}{1.cm}
\setlength{\headsep}{.1cm}
\setlength{\textheight}{.8\paperheight}
\setlength{\textwidth}{.8\paperwidth}
\setlength{\oddsidemargin}{-.5cm}
\setlength{\evensidemargin}{0.0cm}
\setlength{\marginparwidth}{0.0cm}
\setlength{\marginparsep}{0.0cm}

\definecolor{light-gray}{gray}{0.95}
\definecolor{blue}{HTML}{4169E1}
\definecolor{red}{HTML}{DC143C}
\definecolor{green}{HTML}{2E8B57}
\definecolor{black}{HTML}{000000}
\definecolor{g1}{HTML}{A9A9A9}
\definecolor{g2}{HTML}{696969}
\definecolor{g3}{HTML}{7F7F7F}
\definecolor{g4}{HTML}{D3D3D3}

\newcounter{bsp}[section]\setcounter{bsp}{0}
\renewcommand{\thebsp}{\arabic{section}.\arabic{bsp}}

\newenvironment{calc}[2][]{%
\refstepcounter{bsp}%
\ifstrempty{#1}%
{\mdfsetup{%
frametitle={%
\tikz[baseline=(current bounding box.east),outer sep=0pt]
\node[anchor=east,rectangle,fill=green]
{\strut\scriptsize calculation flow~};}}
}%
{\mdfsetup{%
frametitle={%
\tikz[baseline=(current bounding box.east),outer sep=0pt]
\node[anchor=east,rectangle,fill=green]
{\strut\scriptsize calculation flow~#1};}}%
}%
\mdfsetup{innertopmargin=10pt,innerbottommargin=10pt,linecolor=green,%
linewidth=2pt,topline=true,%
frametitleaboveskip=\dimexpr-\ht\strutbox\relax
}
\begin{mdframed}[]\relax%
\label{#2}}{\end{mdframed}}

\newenvironment{code}[2][]{%
\refstepcounter{bsp}%
\ifstrempty{#1}%
{\mdfsetup{%
frametitle={%
\tikz[baseline=(current bounding box.east),outer sep=0pt]
\node[anchor=east,rectangle,fill=red]
{\strut\scriptsize \small{010101}~};}}
}%
{\mdfsetup{%
frametitle={%
\tikz[baseline=(current bounding box.east),outer sep=0pt]
\node[anchor=east,rectangle,fill=red]
{\strut\scriptsize \small{010101}~#1};}}%
}%
\mdfsetup{innertopmargin=10pt,innertopmargin=10pt,linecolor=red,%
linewidth=2pt,topline=true,%
frametitleaboveskip=\dimexpr-\ht\strutbox\relax
}
\begin{mdframed}[]\relax%
\label{#2}}{\end{mdframed}}

\newcommand{\xdownarrow}[1]{%
  {\left\downarrow\vbox to #1{}\right.\kern-\nulldelimiterspace}
}

\newcommand*{\mprime}{^{\prime}\mkern-1.2mu}
\newcommand*{\mdprime}{^{\prime\prime}\mkern-1.2mu}
\newcommand*{\mtprime}{^{\prime\prime\prime}\mkern-1.2mu}
\newcommand*{\ethr}{e_\text{th}}
\newcommand{\black}[1]{\textcolor{black}{#1}}
\newcommand{\red}[1]{\textcolor{red}{#1}}
\newcommand{\blue}[1]{\textcolor{blue}{#1}}
\newcommand{\green}[1]{\textcolor{green}{#1}}
\newcommand{\caf}{\text{\cjRL{b}}}
\newcommand{\he}{\ket{^3\text{He}}}
\newcommand{\fin}{\hat{\mathcal{O}}_{pq}\otimes\ket{^3\text{He}}}
\newcommand{\hes}{${}^3$He}
\newcommand{\tr}{${}^3$H}
\newcommand{\ls}{\ve{L}\cdot\ve{S}}
\newcommand{\eps}{\epsilon}
\newcommand{\cfg}{\texttt{cfg}}
\newcommand{\bv}{\texttt{bv}}
\newcommand{\as}{a_s}
\newcommand{\at}{a_t}
\newcommand{\ecm}{E_\textrm{\small c.m.}}
\newcommand{\dq}{\mbox{d\hspace{-.55em}$^-$}}
\newcommand{\mpis}{$m_\pi=137~${\small MeV}}
\newcommand{\mpim}{$m_\pi=450~${\small MeV}}
\newcommand{\mpil}{$m_\pi=806~${\small MeV}}
\newcommand{\muh}{\mu_{^3\text{\scriptsize He}}}
\newcommand{\mut}{\mu_{^3\text{\scriptsize H}}}
\newcommand{\mud}{\mu_\text{\scriptsize D}}
\newcommand{\pode}{\beta_{\text{\scriptsize D},\pm1}}
\newcommand{\poh}{\beta_{^3\text{\scriptsize He}}}
\newcommand{\pot}{\beta_{^3\text{\scriptsize H}}}
\newcommand{\com}[1]{{\scriptsize \sffamily \bfseries \color{red}{#1}}}
\newcommand{\eg}{\textit{e.g.}\;}
\newcommand{\ie}{\textit{i.e.}\;}
\newcommand{\cf}{\textit{cf.}\;}
\newcommand{\be}{\begin{equation}}
\newcommand{\ee}{\end{equation}}
\newcommand{\la}{\label}
\newcommand{\ber}{\begin{eqnarray}}
\newcommand{\eer}{\end{eqnarray}}
\newcommand{\nn}{\nonumber}
\newcommand{\half}{\frac{1}{2}}
\newcommand{\thalf}{\nicefrac[]{3}{2}}
\newcommand{\bs}[1]{\ensuremath{\boldsymbol{#1}}}
\newcommand{\bea}{\begin{eqnarray}}
\newcommand{\eea}{\end{eqnarray}}
\newcommand{\beq}{\begin{align}}
\newcommand{\eeq}{\end{align}}
\newcommand{\bk}{\bs k}
\newcommand{\bt}{B_{^{3}\text{H}}}
\newcommand{\bh}{B_{^{3}\text{He}}}
\newcommand{\bd}{B_\text{D}}
\newcommand{\ba}{B_\alpha}
\newcommand{\rgm}{$\mathbb{R}$GM}
\newcommand{\ev}[1] {|\bra #1  \ket |^2}
\newcommand{\parg}[1] {\paragraph*{-\,\textit{#1}\,-}}
\newcommand{\nopi}{\pi\hspace{-6pt}/}
\newcommand{\ve}[1]{\ensuremath{\boldsymbol{#1}}}
\newcommand{\xvec}{\bs{x}}
\newcommand{\rvec}{\bs{r}}
\newcommand{\sgve}{\ensuremath{\boldsymbol{\sigma}}}
\newcommand{\tave}{\ensuremath{\boldsymbol{\tau}}}
\newcommand{\na}{\nabla}
\newcommand{\bra}[1] {\left\langle~#1~\right|}
\newcommand{\ket}[1] {\left|~#1~\right\rangle}
\newcommand{\bet}[1] {\left|#1\right|}
\newcommand{\overlap}[2] {\left\langle\,#1\,\left|\,#2\,\right.\right\rangle}
\newcommand{\me}[3] {\left\langle\,#1\,\left|\left.\,#2\,\right|\,#3\,\right.\right\rangle}
\newcommand{\redme}[3] {\left\langle\,#1\,\middle|\right|\,#2\,\left|\middle|\,#3\,\right\rangle}
\newcommand{\lam}[1]{$\Lambda=#1~$fm$^{-1}$}\newcommand{\tx}{\tilde{x}}
\newcommand{\eftnopi}{\mbox{EFT($\slashed{\pi}$)}}
\newcommand{\threej}[6]{\ensuremath{\begin{pmatrix}#1 & #2 & #3\\#4&#5&#6 \end{pmatrix}}}
\newcommand{\clg}[6]{\ensuremath{\left\langle\left.\,#1#2\,;\,#3#4\,\right\vert\right.\left.\,#5#6\,\right\rangle}}
\newcommand{\re}[1] {\mathcal{R}\left[#1\right]}
\newcommand{\im}[1] {\mathcal{I}\left[#1\right]}
\newcommand{\E}{\mathcal{E}}
\newcommand{\eqr}[1]{eq.~\eqref{#1}}
\newcommand{\figref}[1]{fig.~\ref{#1}}
\newcommand{\tabref}[1]{table~\ref{#1}}
\newcommand{\ccite}[1]{ref.~\cite{#1}}
\newcommand{\op}[1] {\hat{\mathcal{O}}_{\text{\small #1}}}
\newcommand{\fkt}[1]{\colorbox{light-gray}{\texttt{#1}}}
\newcommand{\var}[1]{\small\textcolor{blue}{\texttt{#1}}}
\newcommand{\comm}[1]{\small\textcolor{green}{\texttt{#1}}}
\newcommand{\ham}{\mathbb{H}}
\newcommand{\nor}{\mathbb{N}}

\usepackage{pifont}
\renewcommand\thefootnote{\ding{\numexpr171+\value{footnote}}}
\let\endtitlepage\relax


\begin{document}

\title{Lorentz and Siegert offer Compton their assistance.}
\author{Johannes Kirscher}
\email{kirschjs@web.de}
\affiliation{Starfleet Academy, Fort Baker, San Francisco, Earth}
\date{\today}

\begin{abstract}

\end{abstract}

\paragraph{}
\dotfill\\
\begin{calc}[]{calc:ess}
\begin{enumerate}
\item \comm{git clone git@github.com:kirschjs/LIT\char`_source.git}
\item \comm{make all} in \fkt{./LIT\char`_source/src\char`_nucl/}, \fkt{./LIT\char`_source/src\char`_nucl/V18\char`_PAR/},\\
 \fkt{./LIT\char`_source/src\char`_nucl/UIX\char`_PAR/}, and \fkt{./LIT\char`_source/src\char`_elma\char`_pol/}
\item \comm{cd ./LIT\char`_source/src\char`_python/EugenicistsApproach}
\item \comm{./python3.x NextToNewestGeneration.py}
\item \comm{./python3.x A3\char`_lit\char`_M.py}
\item \comm{mathematica ./LIT\char`_source/src\char`_mathematica/helion\char`_E1\char`_multiBas\char`_crosssection.nb}
\item set \var{resultsDir} in 3rd cell to the directory specified by the last output of \fkt{A3\char`_lit\char`_M.py}
\end{enumerate}
\vspace{1cm}
support: \texttt{kirscjo@gmail.com} or \comm{skype} to \texttt{kirschjs.dc}~; 
\end{calc}

\newpage

code repository:~\comm{git clone git@github.com:kirschjs/LIT\char`_source.git}
\begin{enumerate}
\item optimize bases for final and initial states

\begin{calc}[step (4)]{calc:steps}
\begin{tabularx}{0.9\textwidth}{l|X}
\multicolumn{2}{c}{\hrulefill\fkt{NextToNewestGeneration.py}\hrulefill} \\
{argument} & {comment} \\\hline\hline
\var{bastypes} & see \fkt{bridgeA3.py} for the defined structures, 
\newline 
\eg, $\he=\textsf{npp0.5+}$ (\var{channels} dictionary)\\
\var{anzStreuBases} & nbr. of bases grown from different initial seed bases\\
\var{CgfCycles} & nbr. of cycles each cfg. $\in$ \var{bastypes} is optimized\\
\var{nRaces} & nbr. of generations for each \var{CgfCycle}\\
\var{cradleCapacity} & nbr. of children produces within a \var{Race}\\
\var{ini\char`_grid\char`_bounds} & arg. of function \fkt{seedMat}; bounds of the initial,
\newline
 partially randomized geometric grid;
 \newline
  8-element array\\
\var{ini\char`_dims} & arg. of function \fkt{seedMat}; nbr. of width
\newline
 parameters per cfg. for Jacobi 1,2 ($\gamma_{1,2}$);
 \newline 1st pair: initial state; 2nd pair: final state\\
\var{minCond} & minimal condition number$=$ratio between
\newline
 absolute values of the smallest and largest norm eigenvalue;\\
\var{denseEVinterval} & parameter to det. loveliness of a vector in\newline
\fkt{loveliness} and \fkt{basQ} in \fkt{genetic\char`_width\char`_growth.py}\\
\var{removalGainFactor} & while stabilizing the initial seed basis and purging
\newline
 an optimized basis after \var{CfgCycles} of idlers,\newline
  the removal of one of the latter must increase the quality by this factor\\
\var{maxOnPurge} & max. nbr. of basis vectors tested for their effect\newline on stability; ideally $=$ \textsf{dim}(basis)\\
\var{maxOnTrail} & max. nbr. of basis vectors tested for their effect\newline on quality; ideally $=$ \textsf{dim}(basis)\\
\var{muta\char`_initial} & mutation rate (random bit flip) during\newline offspring generation\\\hline
%
output & written in \texttt{respath} (set in \fkt{bridgeA3.py})\\\hline\hline
\var{Ssigbasv3heLIT\char`_Jpi\char`_BasNR-$<$basisSet$>$.dat} & FORTRAN bookkeeping\\
\var{SLITbas\char`_full\char`_Jpi\char`_BasNR-rndSet .dat} & *\\
\var{Ssigbasv3heLIT\char`_Jpi .dat} & *\\
\var{SLITbas\char`_full\char`_Jpi .dat} & *\\
\var{mat\char`_Jpi\char`_BasNR-rndSet} & 
Norm: $\nor=\overlap{\Phi_i}{\Phi_j}/\sqrt{\overlap{\Phi_i}{\Phi_i}\cdot\overlap{\Phi_j}{\Phi_j}}$\newline
Hamiltonian: $\ham=\me{\Phi_i}{\hat{H}}{\Phi_j}/\sqrt{\overlap{\Phi_i}{\Phi_i}\cdot\overlap{\Phi_j}{\Phi_j}}$;
  \newline
   $\ham$ is specified in \fkt{bridgeA3.py} with \var{potnn(n)}, \var{tnni} \newline
   $\nor$ is normalized such that its diagonal $=\mathbb{1}$;\newline
\textcolor{red}{ECCE units:} $\left[\nor_{ij}\right]=0$\newline
 $\left[\ham_{ij}\right]=\left[\hat{H}\right]=\textsf{MeV}$
\\
\hline\hline
\end{tabularx}
\end{calc}

\newpage

\item calculate the overlap matrix elements

\begin{calc}[step (5)]{calc:steps}
\begin{tabularx}{0.9\textwidth}{l|X}
\multicolumn{2}{c}{\hrulefill\fkt{A3\char`_lit\char`_M.py}\hrulefill} \\
{argument} & {comment} \\\hline\hline
\var{multipolarity} $L$ & $j_L(kr_i)Y_{LM}(\hat{\ve{r}}_i)$ 
\newline 
with photon energy $k$ ()\\
\hline
%
output & written in \texttt{respath} (set in \fkt{bridgeA3.py})\\\hline\hline
\var{componentN\char`_S\char`_finalJ\char`_finalMJ\char`_multipoleM} & \\
\hline\hline
\end{tabularx}
\end{calc}

\item 
\end{enumerate}

\paragraph{Quantity of interest}
\dotfill\\
The result of the following steps is the part of the Compton-scattering amplitude (eq.~(2) in~\ccite{Bampa})

\be\la{eq.cptA}
T
_
{\lambda\mprime\lambda}
^
{if}\left(\ve{k}\mprime,\ve{k}\right)
\;\;
\text{with}
\;\;
\left\lbrace
\begin{array}{ll}
\lambda\mprime(\lambda) & \text{out-(in-)going photon polarization}~ \ve{e}_{\lambda}\\
\ve{k}\mprime(\ve{k}) & \text{out-(in-)going photon 3-momentum}\\
i(f) & \text{quantum numbers of the initial(final) nuclear target}
\end{array}
\right.
\;;\quad
\ee
with an intermediate, zero-photon state with energy $\omega+E_i$.

\paragraph{Basis optimization}
\dotfill\\
Initial and final nuclear states are expanded in a non-orthonormal basis. The expansion should
approximate physical features of these states relevant at the scales at which the amplitude is
to be calculated. This scale determines the coupling mechanism to the electromagnetic field,
and demands a more or less accurate description of the nuclear state. But this is a different story,
while the details of

\begin{gather}\la{eq.ifexp}
\overlap{\underline{\ve{\rho}}}{J^\pi}
=
\sum_n~
c_n~
\Phi_{n,\pi,[L_n\otimes S_n]^J}^{\text{RGM}}
\;\;
\text{with}
\;\;
\left\lbrace
\begin{array}{ll}
[L_n\otimes S_n]^J & \text{\ie, LS-coupling scheme}\\
\underline{\ve{\rho}} & \text{set of $A-1$ 3-d Jacobi vectors}\\
J,\pi & \text{total angular momentum, parity of the state}
\end{array}
\right.
\;;\quad
\intertext{and a decomposition which couples orbital- and spin-angular momenta }
\Phi_{n,\pi,[L_n\otimes S_n]^J}^{\text{RGM}}
=
\left[\phi_n\otimes\Xi_n\right]^J\cdot\mathcal{T}_n
\end{gather}
are to be set, here. The orbital part, $\phi$ of $\Phi$ is

\be\la{eq.RGMBV} 
\phi_n= \prod_{i=1}^{A-1}e^{-\gamma_{n,i}\ve{\rho}_i^2}
\cdot
\mathcal{Y}_{l_{n,i}m}(\ve{\rho}_i)
\ee

\begin{code}[]{bsp:helion}
%\begin{table}
\begin{tabularx}{1.\textwidth}{c|l|X}
    {basis parameter} & {implementation} & {comment} \\ \hline
    $S_n$ &  & {total spin {\bf and} all intermediate couplings} \\
      -   & \fkt{bridgeA3.py} & {spin-orbital configuration combinations considered in $\he$ and $\fin$} \\
      -   & \fkt{three\char`_particle\char`_functions.py} & dictionary \var{elem\char`_spin\char`_prods\char`_3} translates, \eg, \var{he\char`_no1} $\to \left[\left[\sigma_1\otimes\sigma_2\right]^{s_{12}=1}\otimes\sigma_3\right]^{S=\nicefrac{1}{2}}\cdot\left[\left[\tau_1\otimes\tau_2\right]^{\tau_{12}=0}\otimes\tau_3\right]^{S=\nicefrac{1}{2}}$
    \\ \hline
\end{tabularx}
%\end{table}
\end{code}

\nocite{*}
\bibliography{refs_compton}

\end{document}